\documentclass{article}
\usepackage{ctex}
\usepackage{amsfonts}
\setCJKmainfont[BoldFont={FZCKJW.TTF}]{STKAITI.TTF}
\usepackage{geometry}
\geometry{a4paper,scale=0.8}

\title{\textbf{\zihao{2}仿生人代替人类}\\\zihao{3}伦理原则冲突分析}
\author{王凯灵}
\date{2021年10月26日}

\begin{document}
	\maketitle
	首先,针对小说本身,我认为,实现小说中复制模仿人类行为的智能模型已经存在,而小说中这类仿生人成为现实的主要困难在于仿生材料和芯片算力。但是,我们所发展的仿真人应该能够在一定程度上利用随机算法做出新的判断,即使不是高级智能,也应该具有一定的推演能力,并且不断接近人类涉及人格的判断模式。这种仿生人之后略做讨论。我暂且就文章中的仿生人展开讨论。我认为,文中的仿生人代替亲人违反了大量的伦理原则。

	首先,文中仿生人违反了IEEE原则的人权原则。文中叙述,仿生人已经能够完全模仿人类,那么如果仿生人在外做出了一些不当的行为,被他人看见,会不会损害过世亲人的名誉?仿生人妞妞长不大的身体周而复始,使得邻居和保安对她有怪异眼光,他们难免对妞妞本人也有不好印象。过世亲人自己是否同意自己的肖像和其他身体数据,以及包括自己行为模式在内的各类个人隐私被使用,又是一个问题。此外,如果我们给予仿生人基本权力,比如选举权,那么仿生人是否剥夺了逝者的人权?但若我们不给仿生人权力,我们是否又剥夺了逝者的人权?如果我对妞妞仿生人做一些不可描述的事情,不就和deep fake将女明星做成色情片性质相当,甚至更加恶劣了吗。而且,仿生人也违反了IEEE的负责任原则。仿生人从地下产业问世,制造人不得而知,责任链完全无从构建。

	其次,仿生人也违反了阿西莫夫原则。第一,妞妞仿生人从窗户掉下,这直接导致了沈兰的死。在类似的情况中,看到一个小孩有生命危险,像我这样的正义青年,即使只是路过,也会本能地救她。虽然仿生人是无心的,但仿生人的行为对于人类的诱导确实导致了人类的伤亡。第二,文中妞妞是婴孩的仿生人,显然她并不能按照董方和沈兰的指令行动,并且由于仿生人模仿婴孩的特性,她还会一直违反人类指令。第三,显然,妞妞并不能保护自己。由于对妞妞的完全复刻,仿生人甚至还几乎是主动地从楼上掉下损伤自己。

	我认为,文中这类仿生人依靠行为的排列组合仿生,没有任何的实际意义。可以说,妞妞仿生人的用途完全只是作用于夫妻二人的心理。当然妞妞仅仅只是作为夫妻的心里依托,是妞妞本人的纯代替。利用人类对亲人的情感倾向,如果利用这类仿生人进行宣传,很容易在潜移默化中影响人的思维。尤其是家中有孩子的,整个价值观都会受到仿生人的影响。那么如果是诈骗行业挂钩仿生人,结果可想而知。所以我提出,仿生人应该遵守不对人类进行心理诱导或思想渗透的原则。妞妞仿生人显然违反了。文中,妞妞仿生人逐渐使沈兰的内心变得扭曲,并且进一步导致了董方忍无可忍,进一步在沈兰死后崩溃。

	屏蔽了上一原则,实际上仿生人已经很难真正像人类了。仿生人要么需要警言慎行,要么要有明显的仿生人区分标签。但这实际上才是更加实际的仿生人发展方向。恐怖谷理论指出,当机器人与人类十分相似,但又不完全一致时,会导致人类恐惧。试想你突然发现和你谈笑风生的陌生人突然停电了。那确实是一件很恐怖的事,有影响人类心理健康的可能。对于妞妞也是这样。如果妞妞在和同龄小朋友一起玩耍同时停电,我想其他小朋友一辈子也不会忘记那个场景。

	但是如果人类越过了恐怖谷,有朝一日。制造出了与人类完全一致,没有半点区别的仿生人。这意味着,仿生人完全能够自理,和人脑一样自主思考。考虑这类仿生人与低等动物的重要性:如果仿生人更重要,那么人造物具有了相当于生命的价值,我们应当逐渐赋予仿生人和人类近乎平等的权利,这就又涉及机器和人类地位问题。稍有不当,过则机器取代人,不则产生人对智慧“生命”的奴役;如果低等动物更重要,那就是人类对于自己高级智慧的否定了。轨道问题,一边是一个完全仿真的妞妞,另一边是小狗,救谁也不是。我想,这会成为一个历史性的伦理问题,超越克隆人类、人类基因编辑问题。

	回到文中的仿生人,我再有一条原则:仿生人不能挑战人的地位。针对仿生人替代亲人这一点,那你人都能被替代了,仿生人养起来还方便,虽然贵,但说不定也没有亲人医药费贵。万一形成了“你死就死呗,我买个仿生的代替”的社会思想,似乎人类存在的意义都遭到了挑战。

	用仿生人取代过世亲人,可行性还有待讨论。但是其确实违反的伦理原则,产生的众多伦理问题必将困扰人类很久。
\end{document}
